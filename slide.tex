%% This Beamer template is based on the one found here: https://github.com/sanhacheong/stanford-beamer-presentation, and edited to be used for Stanford ARM Lab

\documentclass[10pt]{beamer}
%\mode<presentation>{}

\usepackage{media9}
\usepackage{amssymb,amsmath,amsthm,enumerate}
\usepackage{mathtools}
\usepackage[utf8]{inputenc}
\usepackage{array}
\usepackage[parfill]{parskip}
\usepackage[utf8]{vietnam}
\usepackage{graphicx,animate}
\usepackage{caption}
\usepackage{subcaption}
\usepackage{bm}
\usepackage{amsfonts,amscd}
\usepackage[]{units}
\usepackage{listings}
\usepackage{multicol}
\usepackage{multirow}
\usepackage{tcolorbox}
\usepackage{physics}
\usepackage{movie15}
% Enable colored hyperlinks
\hypersetup{colorlinks=true}

% The following three lines are for crossmarks & checkmarks
\usepackage{pifont}% http://ctan.org/pkg/pifont
\newcommand{\cmark}{\ding{51}}%
\newcommand{\xmark}{\ding{55}}%

% Numbered captions of tables, pictures, etc.
\setbeamertemplate{caption}[numbered]
\usepackage{media9} 
%\usepackage[superscript,biblabel]{cite}
%\usepackage{algorithmic}
%\usepackage{algorithm2e}
%\usepackage{algpseudocode}
\usepackage[linesnumbered,ruled,vlined]{algorithm2e}
%\usepackage{algorithm}
%\usepackage{algorithmic}
\usepackage{caption}
%\usepackage{xcolor}
\usepackage{array}
%\renewcommand{\thealgocf}{}

\usepackage[natbib,backend=biber,style=ieee, sorting=ynt]{biblatex}
\bibliography{ref.bib}

\usepackage[acronym]{glossaries}

\usepackage{graphicx}
\graphicspath{{./figures}}
\usepackage{hyperref}

\theoremstyle{remark}
\newtheorem*{remark}{Remark}
\theoremstyle{definition}

%\newcommand{\empy}[1]{{\color{darkorange}\emph{#1}}}
%\newcommand{\empr}[1]{{\color{cardinalred}\emph{#1}}}
%\newcommand{\examplebox}[2]{
%\begin{tcolorbox}[colframe=darkcardinal,colback=boxgray,title=#1]
%#2
%\end{tcolorbox}}

%\usetheme{Stanford} 
%\input{./style_files_stanford/my_beamer_defs.sty}
\usetheme{Copenhagen}
\usecolortheme{seahorse}
\logo{\includegraphics[height=0.5in]{logos/HUS-name.jpg}}

\makeatletter
\let\@@magyar@captionfix\relax
\makeatother

\title[Trực quan hóa dữ liệu]{Trực quan hóa văn bản và tài liệu}

\AtBeginSection[]
{
    \begin{frame}
        \frametitle{Nội dung}
        \tableofcontents[currentsection, subsectionstyle=show/show/hide]
    \end{frame}
}

\setbeamertemplate{page number in head/foot}[totalframenumber]
\setbeamertemplate{frametitle continuation}{}

\begin{document}
\nocite{*}

\author[Nguyễn Chí Thanh - 21007925]{
	\begin{tabular}{c} 
	\Large
	Nguyễn Chí Thanh \\
    \footnotesize \href{mailto:nguyenchithanh\_sdh21@hus.edu.vn}{nguyenchithanh\_sdh21@hus.edu.vn}
\end{tabular}
\vspace{-4ex}}

\institute{
	\vskip 5pt
	\begin{figure}
		\centering
		\begin{subfigure}[t]{0.5\textwidth}
			\centering
			\includegraphics[height=0.75in]{logos/HUS-logo.jpg}
		\end{subfigure}%
		~ 
		\begin{subfigure}[t]{0.5\textwidth}
			\centering
			\includegraphics[height=0.75in]{logos/MIM-logo.png}
		\end{subfigure}
	\end{figure}
	\vskip 5pt	
	Đại học Quốc Gia Hà Nội \\
	Trường đại học Khoa học tự nhiên\\
	Khoa Toán - Cơ - Tin học
	\vskip 3pt
}

%\begin{noheadline}
\begin{frame} \maketitle \end{frame}
%\end{noheadline}
    
\setbeamertemplate{itemize items}[default]
\setbeamertemplate{itemize subitem}[circle]

\begin{frame}{Nội dung}
    \tableofcontents[hidesubsections]
\end{frame}

\section{Các cấp biểu diễn văn bản}

\subsection{Cấp độ từ vựng}

\begin{frame}{Cấp độ từ vựng}
    
\end{frame}

\subsection{Cấp độ cú pháp}

\subsection{Cấp độ ngữ nghĩa}

\section{Mô hình không gian vector}

\subsection{Tính toán các trọng số}

\subsection{Các nhiệm vụ sử dụng mô hình không gian vector}

\section{Trực quan hóa tài liệu đơn}

\subsection{Đám mây từ (Wordcloud)}

\subsection{WordTree}

\subsection{TextArc}

\subsection{Sơ đồ vòng cung (Arc Diagrams)}

\subsection{Dấu ấn văn học (Literature Fingerprinting)}

\section{Trực quan hóa tập tài liệu}

\subsection{Bản đồ tự tồ chức}

\subsection{Hình nền chủ đề}

\subsection{Thẻ tài liệu}

\section{Các kỹ thuật trực quan hóa văn bản mở rộng}

\subsection{Trực quan hóa phần mềm}

\subsection{Trực quan hóa kết quả truy vấn}

\subsection{Trực quan hóa tập tài liệu theo thời gian}

\subsection{Biểu diễn các mối quan hệ}

\section{Tài liệu tham khảo}
\begin{frame}[allowframebreaks]{Tài liệu tham khảo}
    \printbibliography
\end{frame}
\end{document}

